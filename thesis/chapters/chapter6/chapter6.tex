\chapter{Dust Masses in Other Supernova Remnants}\label{chp:chp6}

%\begin{flushright}
%  {\em QUOTE GOES HERE }\\
%
%\ \
%
%\normalsize
%{AUTHOR}  
%\end{flushright}


\section{Introduction}
\noindent{SN~1987A has provided a rare opportunity to follow the evolution of a CCSN in close detail.  It is now well established that there is a significant mass of dust forming in the ejecta.  However, the question of the dust  budget problem in the early universe cannot be solved by considering SN~1987A alone.  If CCSNe are indeed the primary source of the dust seen at high redshifts, it is necessary to establish that the majority of CCSNe do indeed produce sizeable quantities of dust.  This motivates the study of other CCSN remnants with the aim of determining the masses of dust that have formed in their ejecta.}

Blue-shifted line emission is a common and long-lasting feature of the late-time spectra of CCSNe.  In particular the emission lines of oxygen and hydrogen are often visible and often exhibit asymmetries and significant substructure.  If these lines can be modelled then it may be possible to determine the dust masses in SNRs at later times.

There are only a few objects that lend themselves to this sort of modelling however.  The primary obstacle to assessing a large number of late-time spectra from CCSNe is that the majority fade rapidly,  with their brightness decreasing by eight magnitudes after maximum light within the first two years \citep{Kirshner1990}.  They are also frequently further than $\sim$10 Mpc from us and detections are relatively infrequent in the first place.  As a result, the optical spectra are rarely visible after approximately 700 days \citep{Milisavljevic2012}.  


\begin{figure}
\centering
\includegraphics[clip=true,scale=0.6, trim=30 0 50 20]{chapters/chapter6/figs/CasA/spectrum}
\caption{The integrated spectrum of Cas A \citep{Milisavljevic2013}.}
\label{CasA_spectrum}
\end{figure}

However, there are a few unusual objects that, for various reasons, we may still be able to see many years or even decades after maximum light.  This could be because they are unusually close, like SN~1987A, or more usually because some late-time energy source is illuminating the ejecta.  The most obvious of these energy sources is the interaction between the forward shock and the surrounding circumstellar material.  This interaction is known to emit across a wide range of wavelengths from radio to X-rays and cases a reverse shock to propagate inwards through the ejecta heating and ionising the material it passes through.  Other postulated energy sources include interaction between a central pulsar or magnetar and the expanding debris \citep{Woosley2010}, or accretion onto a black hole \citep{Patnaude2001}.

Recent work by \citep{Milisavljevic2012} identified a number of CCSNe that were still visible in the optical despite being more than 20 years old.  These included SN~1957D, SN~1970G, SN~1979C, SN~1980K, SN~1986E, SN~1986J, SN~1987A, SN~1993J and SN~1996cr.  Many of these SNe exhibited strong asymmetries and blue-shifting in their profiles in the optical, particularly in the oxygen and hydrogen lines.  In this chapter I present models of two of these objects, namely SN~1980K and SN~1993J.  I select these as both exhibit at least two blue-shifted line profiles and the lines of interest are largely uncontaminated by other lines with wings that do not overlap with other lines.  

In addition to these SNRs, I also present models of the oxygen lines of Cassiopeia~A (Cas~A).  Cas~A is a very well studied object and at $\sim$300 years old is easily the oldest observable remnant to have detectable line emission in the optical.  I will discuss why this is possible along with some other  interesting facts about Cas~A at the start of the Section \ref{CasA_intro}.  

By modelling these three objects, and from my models of SN~1987A presented in the previous chapter, I hope to be able to determine the order of magnitude of the mass of dust that forms in the ejecta of CCSNe.

\begin{figure}
\centering
\includegraphics[clip=true,scale=0.3, trim=0 0 0 0]{chapters/chapter6/figs/CasA/slit_positions.png}
\caption{Finding charts of the long-slit positions used to compose the integrated spectrum of the main shell of Cas~A.  The background image is a mosaic created from 2004 HST/ACS observations \citep{Fesen2006a}.  Image is taken from \citet{Milisavljevic2013}.}
\label{CasA_slit_positions}
\end{figure}

\section{Cassiopeia A}
\label{CasA_intro}


\subsection{Integrated Spectrum of Cas A}

The integrated spectrum presented in Figure \ref{CasA_spectrum} is from \citet{Milisavljevic2013}.  It is composed of observations from a series of observing runs between September 2007 and November 2010 that were conducted in order to obtain low-dipsersion optical spectra across the remnant.  The majority of observations were carried out at the MDM Observatory at Kitt Peak, Arizona using the 2.4m Hiltner telescope and the Mayall 4m telescope.  The MDM Modular Spectrograph was used with an `Echelle' detector.  A long slit of dimensions 2"$\times$5' was used oriented North-South.  Exposure times were generally 2$\times$500s.  The wavelength range covered was 4500-7000\AA\ with a spectral resolution of 6\AA.  
%A few of the observations of the main shell of Cas~A used alternative spectrographs (the Mark III Spectrograph and the Chivens CCD Spectrograph).  The lowest spectral resolution of any spectrum observed was 12\AA.  

The integrated spectrum was ultimately composed of 80 long slit spectra spaced 3" apart across the entire main shell which is approximately 4' in diameter.  The slit positions are shown in Figure \ref{CasA_slit_positions}.





\begin{figure}
\centering
\includegraphics[scale=0.43,clip=true, trim=30 0 50 20]{chapters/chapter6/figs/CasA/CasA_OIII}
\includegraphics[scale=0.43,clip=true, trim=30 0 50 20]{chapters/chapter6/figs/CasA/CasA_shifted_OIII}
\caption{smooth fits to Cas~A OIII shifted and unshifted}
\label{CasA_OIII}
\end{figure}


\subsection{Smooth Models}

My modelling of the Cas~A spectrum was initially focussed on the [O {\sc iii}] line, which exhibits a pronounced asymmetry.  I managed to produce a reasonable fit to the data using the parameters listed in the first row of Table \ref{CasA_smooth_params}.  The profile is presented in the left pane of Figure \ref{CasA_OIII}.  As can be seen,  the modelled line profile generally fits the observed line profile quite well, although it fails to fit the red side of the profile adequately.  A thorough, manual investigation of parameter space resulted in the conclusion that the profile was much better fitted if the entire observed profile was shifted to the blue by $-700$~km~s$^{-1}$.  This might well be a reasonable assumption.  The integrated spectrum does not sample the entire ejecta.  Additionally, Cas~A is known to be significantly asymmetrical \citep{Rest2011}.  The integrated profile may well sample regions with different line-of-sight velocities, the sum of which could result in a net line-of-sight velocity shift.  This was also found to be the case with the [O {\sc ii}] and [O {\sc i}] lines.  Throughout the remainder of my modelling I therefore allowed the profiles to be shifted in velocity space to better fit the data based on the possibility that the sampled emitting regions had an overall net velocity towards the observer.  Fits to the line profiles were significantly improved following this translation (see Figures \ref{CasA_OIII} to \ref{CasA_OIII_clumped}).

A model of the shifted [O {\sc iii}] line is presented in Figure \ref{CasA_OIII} and the parameters used for this model are presented in the second row of Table \ref{CasA_smooth_params}.  The line profile was shifted by $-700$~km~s$^{-1}$.  A total dust optical depth of $\tau=0.49$ at 5007\AA\ between $R_{in}$ and $R_{out}$ was found to best fit the profile.  An albedo of $\omega\approx0.15$ at 5007\AA\ was also necessary.

\afterpage{
\begin{landscape}
\setlength{\tabcolsep}{10pt}
\begin{table}
\centering
%	\begin{minipage}{180mm}
	\caption{The parameters used for the smooth models of Cas~A with a medium composed of 50\% amorphous carbon and 50\% silicate grains of radius $a=0.05~\mu$m.  Optical depths are given from $R_{in}$ to $R_{out}$ at $\lambda = 5007$~\AA\ for [O~{\sc iii}], $\lambda = 7319$~\AA\ for [O~{\sc ii}] and $\lambda = 6300$~\AA\ for [O~{\sc i}].  The doublet ratio is always the ratio of the stronger line to the weaker line. The asterisk indicates that the parameters listed describe the gas density distribution.  The dust density distribution is the same in all cases (as detailed for the shifted [O~{\sc ii}] doublet in the second row).}
	\label{CasA_smooth_params}
	\centering
  	\begin{tabular}{@{} cccccccccccc @{}}
    	\hline
  &$v$ shift& $V_{max}$ & $V_{min}$ & $R_{in}/R_{out}$ & $\beta$ & $M_{dust}$ & $R_{out}$ & $R_{in}$ & doublet  & $\tau_{\lambda}$   \\
	& (km~s$^{-1}) $& (km~s$^{-1} $)& (km~s$^{-1} $) & & & ($M_{\odot}$) & (10$^{18}$cm) & (10$^{18}$cm) & ratio \\
	\hline
[O~{\sc iii}]  & 0 & 4500 & 1800 & 0.4  & 2.0 & 1.1 & 4.7 & 1.9 & 2.9 & 0.53  \\ \relax
[O~{\sc iii}]  & -700 & 5000 & 2500 & 0.5  & 2.0 & 1.1 & 5.2 & 2.6 & 2.9 & 0.49  \\ \relax
[O~{\sc ii}]*  & -1000 & 5000 & 3250 & 0.65  & 2.0 & 1.1 & 5.2 & 3.4 & 1.23 & 0.21  \\ \relax
[O~{\sc i}]*  & -1000 & 5000 & 3250 & 0.65  & 2.0 & 1.1 & 5.2 & 3.4 & 3.1 & 0.30  \\ 
    \hline
  \end{tabular}

%\end{minipage}
\end{table}
\end{landscape}
}


The composition of the dust has a significant effect on the overall dust mass for this optical depth and albedo.  A line profile model of the [O {\sc iii}] line profile from Cas~A could not be found using 100\% astronomical silicate dust \citep{Draine1984}.  There is little to no scattering wing seen, hence the relatively low value of $\omega$, and therefore relatively small silicate grains are required to reproduce the red side of the profile.  Silicate grains of this size have extremely low absorption efficiencies and therefore the best-fitting optical depth of $\tau=0.49$ corresponds to an implausibly large mass of dust ($>20$M$_{\odot}$) if it is composed entirely of astronomical silicates.

The chemical composition of the dust in the eject of Cas~A is known to be extremely complex \citep{Arendt2014}.  There are many different species of dust grain present in the ejecta, and it is known that silicate dust is present based on emission features observed in the IR region of the spectrum.  However, the presence of a variety of other species is also predicted.  I therefore investigate the dust masses required to fit the profile for varying fractions of silicate and amorphous carbon dust.  In Table \ref{CasA_dust_masses}, I detail the dust masses required to fit the [O {\sc iii}] line profile for different fractions of silicates and amorphous carbon grains for a single grain size.  For each composition I determine the grain radius based on the albedo necessary to fit the profile ($\omega\approx0.15$) and then vary the dust mass to achieve the required optical depth.  The calculated dust masses cover a wide range of values between 0.65 - 6.5M$_{\odot}$.   All of the line profile models listed above adopted intrinsic doublet strengths from the theory as detailed by \citet{Zeippen1987} and \citet{Storey2000}.  


\begin{figure}
\centering
\includegraphics[scale=0.41,clip=true, trim=15 0 40 20]{chapters/chapter6/figs/CasA/CasA_shifted1000_OII}
\includegraphics[scale=0.41,clip=true, trim=15 0 40 20]{chapters/chapter6/figs/CasA/CasA_OI_shifted1000}
\caption{Cas~A OI and OII}
\label{CasA_OI_OII}
\end{figure}


\begin{table}
\centering
%	\begin{minipage}{180mm}
	\caption{The variation in dust mass for a fixed optical depth $\tau_{5007\AA}=0.49$ for the parameters listed in Table \ref{CasA_smooth_params}.}
	\label{CasA_dust_masses}
	\centering
  	\begin{tabulary}{12cm}{C C C C}
    	\hline
	\% silicate  & \% amorphous & grain radius $a$ &  $M_{dust}$ \\
	grains& carbon grains&($\mu$m)&(M$_{\odot}$)\\
		\hline
90	&10	&0.035&	6.5 \\
75	&25	&0.04	&2.5\\
50	&50	&0.045&	1.1\\
25	&75	&0.048&	0.6\\
0	&100	&0.05&	0.37\\
    \hline
  \end{tabulary}


\end{table}

It might be possible  to approximately determine the composition based on the relative optical depths necessary to fit different blue-shifted lines in the spectrum and the wavelength dependence of dust absorption for different compositions.  I therefore considered fitting the blue-shifted [O {\sc ii}] and [O {\sc i}] lines from Cas~A.  Sadly, at the small grain sizes required, there is not significant variation in the absorption efficiencies of either amorphous carbon or astronomical silicates between 5007\AA\ and 7319\AA\ and I cannot therefore determine the composition via this approach.  Additionally, the [O {\sc ii}] and [O {\sc i}] lines are much less sensitive to variations in both distribution and dust mass, partly due to the high frequency of bumpy features observed in these lines which contaminate the intrinsic broad profile.  The best-fitting models for these lines were therefore quite degenerate i.e. there were multiple sets of parameters that resulted in reasonable fits.  

However, it was possible to use these lines to determine the feasibility of the best-fitting model for the  [O {\sc iii}] line profile.  I adopted the dust distribution that I determined using the [O {\sc iii}] line and investigated models for the [O {\sc ii}] and [O {\sc i}] line profiles to see if this dust distribution were capable of fitting these lines as well.  I adopted an emissivity distribution that was slightly different to the [O {\sc iii}] line (see Table \ref{CasA_smooth_params}) and shifted the observed line profiles by $-1000$~km~s$^{-1}$.  These emissivity distributions were modelled with the dust distribution and mass for the best-fitting smooth [O {\sc iii}] model.  The resultant [O {\sc ii}] and [O {\sc i}] line profiles are very good fits (see Figure \ref{CasA_OI_OII}).  This suggests that the models are consistent and, if the relative abundance of species can be determined, that the dust masses can be well-constrained.

\subsection{Clumped Models of Cas~A}

The ejecta of Cas~A is highly clumped.  Recently, models by \citet{Biscaro2014} have suggested that dust cannot in fact form in the gas phase in the ejecta of Cas~A unless extremely dense knots of material are present.  It is therefore important, as with SN~1897A, to consider the effects of clumping on the line profiles.  I continue to focus on the [O {\sc iii}] line profile from Cas~A and consider the effects of clumping.  Clearly, the ejecta has a complex geometry with many clumps of different sizes and likely different ionisation states and dust species within each and therefore the models we present here only aim to provide some indication of the effects of clumping within the ejecta.  To this end I present a number of models of the [O {\sc iii}] line profile based on the smooth fits that I presented in the previous section.  I consider two different clump sizes, ones with width $R_{out}/25$ and ones with width $R_{out}/10$.  I also consider three different clump volume filling factors $f=0.05$, $f=0.1$ and $f=0.25$.  For each combination of clump size and filling factor I re-evaluate the required increase or decrease in the dust mass over the smooth model.  All other parameters were kept fixed such that the emission was emitted smoothly according to the distribution and geometry described by the parameters listed in Table \ref{CasA_smooth_params}.

\begin{figure}
\centering
\includegraphics[scale=0.43,clip=true, trim=30 0 50 20]{chapters/chapter6/figs/CasA/clumped/CasA_OIII_c10_f0_05}
\includegraphics[scale=0.43,clip=true, trim=30 0 40 20]{chapters/chapter6/figs/CasA/clumped/CasA_OIII_c25_f0_05}

\vspace{6mm}
\includegraphics[scale=0.43,clip=true, trim=30 0 50 20]{chapters/chapter6/figs/CasA/clumped/CasA_OIII_c10_f0_1}
\includegraphics[scale=0.43,clip=true, trim=30 0 40 20]{chapters/chapter6/figs/CasA/clumped/CasA_OIII_c25_f0_1}

\vspace{6mm}
\includegraphics[scale=0.43,clip=true, trim=30 0 50 20]{chapters/chapter6/figs/CasA/clumped/CasA_OIII_c10_f0_25}
\includegraphics[scale=0.43,clip=true, trim=30 0 40 20]{chapters/chapter6/figs/CasA/clumped/CasA_OIII_c25_f0_25}

\caption{Cas~A clumped}
\label{CasA_OIII_clumped}
\end{figure}

The change in the required dust mass is listed as a fraction of the smooth dust mass (e.g. $M_{dust}=1.1$M$_{\odot}$ for a medium of 50\% astronomical silicates and 50\% amorphous carbon - see Table \ref{CasA_dust_masses} for other dust masses with different dust compositions) is given in Table \ref{CasA_clumped_dust_masses}.  Whilst clumping can be seen to increase the required dust mass in all cases, in the most extreme it is still only by a factor of 3.5.  The fits for each of these cases are presented in Figure \ref{CasA_OIII_clumped}.

\begin{table}
\caption{The fraction of increase in dust mass over the smooth model with parameters as given in Table \ref{CasA_smooth_params} for clumped models with different clump widths and different clump volume filling factors.  The other parameters in the models were fixed at the values given in Table \ref{CasA_smooth_params}.}
\centering
\begin{tabular}{l  c c c}
\hline
& $f=0.05$ &$f=0.1$&$f=0.25$\\
\hline
$R_{out}/10$ & 3.5 & 1.9 & 1.4 \\
$R_{out}/25$ & 1.6 & 1.0 & 1.0 \\
\hline
\end{tabular}
\label{CasA_clumped_dust_masses}
\end{table}

\subsection{The Geometry of Cas~A}

Consider section 2 of conclusions of DAN and FESEN - talk about overall velocity of optical emitting regions being towards us

\subsection{The Chemistry of Cas~A}

Compare to the biscaro paper and the other species paper

\subsection{The Dust Mass in Cas~A}

Mike's paper and others.  Need a reason why my estimate is higher.



\clearpage

\section{SN~1980K}

\subsection{Smooth Models}
\begin{figure}
\centering
\includegraphics[scale=0.4,clip=true, trim=20 0 40 20]{chapters/chapter6/figs/80K/smooth/Ha}
\includegraphics[scale=0.4,clip=true, trim=20 0 40 20]{chapters/chapter6/figs/80K/smooth/Ha_amC}

\includegraphics[scale=0.4,clip=true, trim=20 0 40 20]{chapters/chapter6/figs/80K/smooth/OI}
\includegraphics[scale=0.4,clip=true, trim=20 0 40 20]{chapters/chapter6/figs/80K/smooth/OI_amC}
\caption{Smooth fits to SN 1980K}
\label{80K_smooth}
\end{figure}

\begin{figure}
\centering
\includegraphics[scale=0.4,clip=true, trim=20 0 40 20]{chapters/chapter6/figs/80K/clumped/Ha}
\includegraphics[scale=0.4,clip=true, trim=20 0 40 20]{chapters/chapter6/figs/80K/clumped/Ha_amC}

\includegraphics[scale=0.4,clip=true, trim=20 0 40 20]{chapters/chapter6/figs/80K/clumped/OI}
\includegraphics[scale=0.4,clip=true, trim=20 0 40 20]{chapters/chapter6/figs/80K/clumped/OI_amC}
\caption{clumped fits to SN 1980K}
\label{80K_clumped}
\end{figure}

\clearpage

\section{SN~1993J}
\begin{figure}
\centering
\includegraphics[scale=0.4,clip=true, trim=20 0 40 20]{chapters/chapter6/figs/93J/smooth/OIII}
\includegraphics[scale=0.4,clip=true, trim=20 0 40 20]{chapters/chapter6/figs/93J/smooth/OIII_amC}

\includegraphics[scale=0.4,clip=true, trim=20 0 40 20]{chapters/chapter6/figs/93J/smooth/OII}
\includegraphics[scale=0.4,clip=true, trim=20 0 40 20]{chapters/chapter6/figs/93J/smooth/OII_amC}
\caption{smooth fits to SN 1993J}
\label{93J_smooth}
\end{figure}









\section{Discussion}

\begin{figure}
\centering
\fbox{\includegraphics[scale=0.6,clip=true, trim=30 0 0 0]{chapters/chapter6/figs/test.png}}
\caption{dust masses}
\label{shifted}
\end{figure}


\section{Conclusions}
