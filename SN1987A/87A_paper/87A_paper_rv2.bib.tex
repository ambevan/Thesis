Automatically generated by Mendeley Desktop 1.14
Any changes to this file will be lost if it is regenerated by Mendeley.

BibTeX export options can be customized via Preferences -> BibTeX in Mendeley Desktop

@article{Hammer2009,
abstract = {We present the first three-dimensional (3D) simulations of the large-scale mixing that takes place in the shock-heated stellar layers ejected in the explosion of a 15.5 solar-mass blue supergiant star. The outgoing supernova shock is followed from its launch by neutrino heating until it breaks out from the stellar surface more than two hours after the core collapse. Violent convective overturn in the post-shock layer causes the explosion to start with significant asphericity, which triggers the growth of Rayleigh-Taylor (RT) instabilities at the composition interfaces of the exploding star. Deep inward mixing of hydrogen (H) is found as well as fast-moving, metal-rich clumps penetrating with high velocities far into the H-envelope of the star as observed, e.g., in the case of SN 1987A. Also individual clumps containing a sizeable fraction of the ejected iron-group elements (up to several 0.001 solar masses) are obtained in some models. The metal core of the progenitor is partially turned over with Ni-dominated fingers overtaking oxygen-rich bullets and both Ni and O moving well ahead of the material from the carbon layer. Comparing with corresponding 2D (axially symmetric) calculations, we determine the growth of the RT fingers to be faster, the deceleration of the dense metal-carrying clumps in the He and H layers to be reduced, the asymptotic clump velocities in the H-shell to be higher (up to \~{}4500 km/s for the considered progenitor and an explosion energy of 10\^{}\{51\} ergs, instead of <2000 km/s in 2D), and the outward radial mixing of heavy elements and inward mixing of hydrogen to be more efficient in 3D than in 2D. We present a simple argument that explains these results as a consequence of the different action of drag forces on moving objects in the two geometries. (abridged)},
archivePrefix = {arXiv},
arxivId = {0908.3474},
author = {Hammer, N. J. and Janka, H. -Th. and Mueller, E.},
doi = {10.1088/0004-637X/714/2/1371},
eprint = {0908.3474},
file = {:Users/antoniabevan/Documents/Mendeley Desktop/Hammer, Janka, Mueller/Unknown/Hammer, Janka, Mueller - 2009 - Three-Dimensional Simulations of Mixing Instabilities in Supernova Explosions.pdf:pdf},
issn = {0004-637X},
keywords = {general,hydrodynamics,instabilities,shock waves,supernovae},
pages = {15},
title = {{Three-Dimensional Simulations of Mixing Instabilities in Supernova Explosions}},
url = {http://arxiv.org/abs/0908.3474},
year = {2009}
}
@article{Gomez2012,
author = {Gomez, H. L. and Clark, C. J. R. and Nozawa, T. and Krause, O. and Gomez, E. L. and Matsuura, M. and Barlow, M. J. and Besel, M.-a. and Dunne, L. and Gear, W. K. and Hargrave, P. and Henning, Th. and Ivison, R. J. and Sibthorpe, B. and Swinyard, B. M. and Wesson, R.},
doi = {10.1111/j.1365-2966.2011.20272.x},
file = {:Users/antoniabevan/Documents/Mendeley Desktop/Gomez et al/\mnras/Gomez et al. - 2012 - Dust in historical Galactic Type Ia supernova remnants with Herschel★.pdf:pdf},
issn = {00358711},
journal = {\mnras},
keywords = {abundances,and with,dust,european-led principal investigator consortia,extinc-,galaxies,herschel is an european,important participation from nasa,individual,ism,kepler,s space observatory with,science instruments provided by,space agency,submillimetre,supernovae,tion,tycho},
month = mar,
number = {4},
pages = {3557--3573},
title = {{Dust in historical Galactic Type Ia supernova remnants with Herschel★}},
url = {http://doi.wiley.com/10.1111/j.1365-2966.2011.20272.x},
volume = {420},
year = {2012}
}
@article{Hillier1991,
abstract = {The importance of electron-scattering wings is examined with respect to the Wolf-Rayet (W-R) stellar spectra, the entire line profile, and the diagnosis of density inhomogeneities in stellar winds. The influence of inhomogeneities on the derived mass-loss rates is then examined to develop ideas related to W-R stellar winds and stellar evolution. Existing evidence regarding electron-scattering wings in W-R stellar spectra is reviewed, and present models are found to be inadequate to describe the weak wings. Theoretical line profiles are computed by two separate methods, and the effect of an inhomogeneous density structure on the strength of the electron-scattering wings is examined. The line profiles and line strengths are similar to observed values except that the mean mass-loss rate is lower; the corresponding reduction in the strength of the wings shows that the mass-loss rates for W-R stars are overestimated.},
author = {Hillier, D. J.},
file = {:Users/antoniabevan/Documents/Mendeley Desktop/Hillier/\aap/Hillier - 1991 - The effects of electron scattering and wind clumping for early emission line stars.pdf:pdf},
journal = {\aap},
pages = {455--468},
title = {{The effects of electron scattering and wind clumping for early emission line stars}},
url = {http://adsabs.harvard.edu/abs/1991A\&A...247..455H},
volume = {247},
year = {1991}
}
@article{Hanuschik1993,
abstract = {We investigate the fine structure of the H$\alpha$ emission line from SN$\backslash$n1987A in the period days 115-673. We find at least five bumps in the$\backslash$nradial velocity range -2000 to + 1300 km s\^{}-1\^{}, with typical widths of$\backslash$n400-500 km s\^{}-1\^{} and amplitudes 7-10 percent above the local emission$\backslash$nlevel. These bumps start showing up on day 115+/-5 and remain visible$\backslash$nuntil day 523, at least one bump still being visible on day 673. Between$\backslash$ndays 115 and 523, the radial velocity of the bumps is approximately$\backslash$nconstant, with their amplitude gradually increasing. Between days 523$\backslash$nand 673, most bumps almost disappear, while one bump increases in$\backslash$nrelative strength. At that epoch, dust forms in the ejecta and changes$\backslash$nthe overall line profile considerably. Assuming that the bumps are due$\backslash$nto local density enhancements (clumps) and that they suffer from dust$\backslash$nextinction to a degree depending on their location in the ejecta, we$\backslash$nshow that the observed intensity contrast changes can be explained by$\backslash$nselective dust extinction alone, if dust is distributed homogeneously in$\backslash$nan inner shell of radius r\_d\_ = 0.6 R\_max\_ (where R\_max\_ is the outer$\backslash$nradius of the hydrogen shell) with dust optical depth $\tau$ = 3.5, and$\backslash$nif the clumps are concentrated in the inner part of the hydrogen shell$\backslash$nat average radius 0.5 R\_max\_.},
author = {Hanuschik, R. W. and Spyromilio, J. and Stathakis, R. and Kimeswenger, S. and Gochermann, J. and Seidensticker, K. J. and Meurer, G.},
file = {:Users/antoniabevan/Documents/Mendeley Desktop/Hanuschik et al/\mnras/Hanuschik et al. - 1993 - Clumps in Supernova 1987A - the H$\alpha$ Line.pdf:pdf},
journal = {\mnras},
pages = {909},
title = {{Clumps in Supernova 1987A - the H$\alpha$ Line}},
url = {http://adsabs.harvard.edu/abs/1993MNRAS.261..909H$\backslash$nhttp://articles.adsabs.harvard.edu/cgi-bin/nph-iarticle\_query?1993MNRAS.261..909H\&amp;data\_type=PDF\_HIGH\&amp;whole\_paper=YES\&amp;type=PRINTER\&amp;filetype=.pdf},
volume = {261},
year = {1993}
}
@article{Sarangi2015,
archivePrefix = {arXiv},
arxivId = {1412.5522v1},
author = {Sarangi, Arkaprabha and Cherchneff, Isabelle},
doi = {10.1051/0004-6361/201424969},
eprint = {1412.5522v1},
file = {:Users/antoniabevan/Documents/Mendeley Desktop/Sarangi, Cherchneff/Astronomy \& Astrophysics/Sarangi, Cherchneff - 2015 - Condensation of dust in the ejecta of Type II-P supernovae.pdf:pdf},
issn = {0004-6361},
journal = {Astronomy \& Astrophysics},
keywords = {astrochemistry,circumstellar matter,dust,ex-,general,ism,molecular processes,supernova remnants,supernovae},
pages = {A95},
title = {{Condensation of dust in the ejecta of Type II-P supernovae}},
url = {http://www.aanda.org/10.1051/0004-6361/201424969},
volume = {575},
year = {2015}
}
@article{Auer1972,
author = {Auer, LH and Blerkom, D Van},
file = {:Users/antoniabevan/Documents/Mendeley Desktop/Auer, Blerkom/\apj/Auer, Blerkom - 1972 - Electron scattering in spherically expanding envelopes.pdf:pdf},
journal = {\apj},
pages = {175--181},
title = {{Electron scattering in spherically expanding envelopes}},
url = {http://scholar.google.com/scholar?hl=en\&btnG=Search\&q=intitle:No+Title\#0 http://adsabs.harvard.edu/full/1972ApJ...178..175A},
volume = {178},
year = {1972}
}
@article{Spyromilio1993,
abstract = {We investigate high-resolution line profiles of H$\alpha$, [Fe II] 7155A$\backslash$nand Ca II] 7291,7323 A in the spectra of SN 1987A. We identify a common$\backslash$nprominent narrow feature in all four lines. We attribute the emission to$\backslash$na fast knot of metals and hydrogen. Our observations provide the first$\backslash$ndirect evidence for mixing on a small as well as a large scale in the$\backslash$nejecta of SN 1987A. Our observations support the idea that such knots$\backslash$nmay well have contributed the bulk of the emission of the supernova$\backslash$nafter the ejecta cooled and dust formed at about day 500. The ratio of$\backslash$nthe two components of the blend of Ca II] 7291,7323A provides an upper$\backslash$nlimit on the number density of Ca\^{}+\^{} of 1 x 10\^{}5\^{} cm\^{}-3\^{} on day 673 for$\backslash$nboth the knot and the rest of the ejecta.},
author = {Spyromilio, J. and Stathakis, R. a. and Meurer, G. R.},
file = {:Users/antoniabevan/Documents/Mendeley Desktop/Spyromilio, Stathakis, Meurer/\mnras/Spyromilio, Stathakis, Meurer - 1993 - Clumping and Smallscale Mixing in Supernova 1987A.pdf:pdf},
journal = {\mnras},
pages = {530},
title = {{Clumping and Smallscale Mixing in Supernova 1987A}},
url = {http://adsabs.harvard.edu/abs/1993MNRAS.263..530S$\backslash$nhttp://articles.adsabs.harvard.edu/cgi-bin/nph-iarticle\_query?1993MNRAS.263..530S\&amp;data\_type=PDF\_HIGH\&amp;whole\_paper=YES\&amp;type=PRINTER\&amp;filetype=.pdf},
volume = {263},
year = {1993}
}
@article{Lucy2005c,
author = {Lucy, LB},
doi = {10.1051/0004-6361},
file = {:Users/antoniabevan/Documents/Mendeley Desktop/Lucy/\aap/Lucy - 2005 - Monte Carlo techniques for time-dependent radiative transfer in 3-D supernovae.pdf:pdf},
journal = {\aap},
keywords = {general,methods,numerical,radiative transfer,stars,supernovae},
pages = {19--30},
title = {{Monte Carlo techniques for time-dependent radiative transfer in 3-D supernovae}},
url = {http://www.aanda.org/articles/aa/abs/2005/01/aa1656/aa1656.html},
volume = {30},
year = {2005}
}
@article{Groningsson2008,
author = {Gr\"{o}ningsson, P. and Fransson, C. and Lundqvist, P. and Lundqvist, N. and Leibundgut, B. and Spyromilio, J. and Chevalier, R. a. and Gilmozzi, R. and Kj\ae r, K. and Mattila, S. and Sollerman, J.},
doi = {10.1051/0004-6361:20077604},
file = {:Users/antoniabevan/Documents/Mendeley Desktop/Gr\"{o}ningsson et al/\aap/Gr\"{o}ningsson et al. - 2008 - High resolution spectroscopy of the inner ring of SN 1987A.pdf:pdf},
issn = {0004-6361},
journal = {\aap},
keywords = {circumstellar matter,individual,shock waves,sn 1987a,supernovae},
number = {3},
pages = {761--777},
title = {{High resolution spectroscopy of the inner ring of SN 1987A }},
volume = {479},
year = {2008}
}
@article{Chevalier1989,
abstract = {Analytic models for the spherically symmetric expansion of SN 1987 A are discussed. The SN model includes a bend in the expanded density profiles at a velocity of about 4000 km/s and employs an asymmetric initial density distribution or energy deposition on the assumption of radial flow. The calculations show that there is a tendency for the flow to become more spherical during the expansion phase, and that the sense of density asymmetry changes across the bend in the density profile.},
author = {Chevalier, Roger A and Soker, Noam},
doi = {10.1086/167545},
file = {:Users/antoniabevan/Documents/Mendeley Desktop/Chevalier, Soker/\apj/Chevalier, Soker - 1989 - Asymmetric envelope expansion of supernova 1987A.pdf:pdf},
issn = {0004-637X},
journal = {\apj},
pages = {867--882},
title = {{Asymmetric envelope expansion of supernova 1987A}},
volume = {341},
year = {1989}
}
@article{Kozma1997,
abstract = {Using the temperature and ionization calculated in our previous paper, we model the spectral evolution of SN 1987A. The IR-catastrophe is seen in the metal lines as a transition from thermal to non-thermal excitation, most clearly in the [O I] 6300, 6364 lines. The distribution of the different zones, and therefore the gamma-ray deposition, is determined from the line profiles of the most important lines, where possible. We find the total mass of hydrogen-rich gas to be \~{}7.7 Msun. The helium mass derived from the line fluxes is sensitive to assumptions about the degree of redistribution in the line. The mass of the helium dominated zone is consistent with \~{}1.9 Msun, with a further \~{}3.9 Msun of helium residing in the hydrogen component. Because of uncertainties in the modeling of the non-thermal excitation of the [O I] lines, the uncertainty in the oxygen mass is considerable. In addition, masses of nitrogen, neon, magnesium, iron and nickel are estimated. The dominant contribution to the line luminosity often originates in a different zone from where most of the newly synthesized material resides. This applies to e.g. carbon, calcium and iron. The [C I] lines, mainly arising in the helium zone, indicate a substantially lower abundance of carbon mixed with helium than stellar evolution models give, and a more extended zone with CNO processed gas is indicated. The [Fe II] lines have in most phases a strong contribution from primordial iron, and at t > 600 - 800 days this component dominates the [Fe II] lines. The wings of the [Fe II] lines may therefore come from primordial iron, rather than synthesized iron mixed to high velocity. Lines from ions with low ionization potential indicate that the UV field below at least 1600 AA is severely quenched by dust absorption and resonance scattering.},
archivePrefix = {arXiv},
arxivId = {astro-ph/9712224},
author = {Kozma, Cecilia and Fransson, Claes},
doi = {10.1086/305452},
eprint = {9712224},
file = {:Users/antoniabevan/Documents/Mendeley Desktop/Kozma, Fransson/\apj/Kozma, Fransson - 1998 - Late Spectral Evolution of SN 1987A II. Line Emission.pdf:pdf},
issn = {0004-637X},
journal = {\apj},
number = {497},
pages = {63},
primaryClass = {astro-ph},
title = {{Late Spectral Evolution of SN 1987A: II. Line Emission}},
url = {http://arxiv.org/abs/astro-ph/9712224},
year = {1998}
}
@article{Chugai2001,
abstract = {I propose that broad narrow-topped emission lines with full width at zero intensity >20000 km/s, seen in early-time spectra of SN 1998S, originate from a dense circumstellar gas and not from the supernova ejecta. The tremendous line width is the result of multiple scattering of the narrow line radiation on thermal electrons of the circumstellar shell with the Thomson optical depth of about 4 on 1998 March 6.},
archivePrefix = {arXiv},
arxivId = {astro-ph/0106234},
author = {Chugai, N. N.},
doi = {10.1111/j.1365-8711.2001.04717.x},
eprint = {0106234},
file = {:Users/antoniabevan/Documents/Mendeley Desktop/Chugai/Unknown/Chugai - 2001 - Broad emission lines from opaque electron-scattering environment of SN 1998S.pdf:pdf},
issn = {0035-8711},
keywords = {circumstellar matter,general,individual,line profiles,scattering,sn 1998s,supernovae},
pages = {7},
primaryClass = {astro-ph},
title = {{Broad emission lines from opaque electron-scattering environment of SN 1998S}},
url = {http://arxiv.org/abs/astro-ph/0106234},
volume = {1454},
year = {2001}
}
@article{Wesson2015,
archivePrefix = {arXiv},
arxivId = {1410.7386v1},
author = {Wesson, R and Barlow, M J and Matsuura, M and Ercolano, B},
eprint = {1410.7386v1},
file = {:Users/antoniabevan/Documents/Mendeley Desktop/Wesson et al/\mnras/Wesson et al. - 2015 - The timing and location of dust formation in the remnant of SN 1987A.pdf:pdf},
journal = {\mnras},
keywords = {clouds,general,individual,ism,magellanic,sn 1987a,supernova remnants,supernovae},
pages = {2089},
title = {{The timing and location of dust formation in the remnant of SN 1987A}},
volume = {446},
year = {2015}
}
@article{Draine1984,
abstract = {The dielectric functions for graphite and astronomical silicate material are computed using available laboratory and astronomical data. It is noted that the magnetic dipole contribution to absorption in the infrared wavelengths can be important for conducting particles such as graphite. Formulas are given for evaluating electric and magnetic dipole cross-sections for small particles. Absorption cross-sections are evaluated for graphite and silicate particles with sizes between 0.003 and 1.0 microns, and wavelengths from 300 A to 1000 microns. On the basis of polarization profiles computed for both prolate and oblate graphite-silicate spheroids, it is concluded that interstellar silicate grain are predominantly oblate. Extinction curves are calculated for Mathis-Rumpl-Nordsieck graphite-silicate grain mixtures and are compared to observations. The model is found to be in good agreement with available infrared observations.},
author = {Draine, B. T. and Lee, H. M.},
doi = {10.1086/162480},
file = {:Users/antoniabevan/Documents/Mendeley Desktop/Draine, Lee/\apj/Draine, Lee - 1984 - Optical properties of interstellar graphite and silicate grains.pdf:pdf},
isbn = {0004-637X},
issn = {0004-637X},
journal = {\apj},
pages = {89--108},
title = {{Optical properties of interstellar graphite and silicate grains}},
volume = {285},
year = {1984}
}
@article{Barlow2010,
author = {Barlow, M. J. and Krause, O. and Swinyard, B. M. and Sibthorpe, B. and Besel, M.-A. and Wesson, R. and Ivison, R. J. and Dunne, L. and Gear, W. K. and Gomez, H. L. and Hargrave, P. C. and Henning, Th. and Leeks, S. J. and Lim, T. L. and Olofsson, G. and Polehampton, E. T.},
doi = {10.1051/0004-6361/201014585},
file = {:Users/antoniabevan/Documents/Mendeley Desktop/Barlow et al/\aap/Barlow et al. - 2010 - A Herschel PACS and SPIRE study of the dust content of the Cassiopeia A supernova remnant.pdf:pdf},
issn = {0004-6361},
journal = {\aap},
keywords = {dust,extinction,infrared,ism,supernova remnants},
month = jul,
pages = {L138},
title = {{A Herschel PACS and SPIRE study of the dust content of the Cassiopeia A supernova remnant}},
url = {http://www.aanda.org/10.1051/0004-6361/201014585},
volume = {518},
year = {2010}
}
@article{Wang1996,
abstract = {We have used the Faint Object Spectrograph on the Hubble Space Telescope to observe the spectra of SN 1987A over the wavelength range 2000 -- 8000$\backslash$ $\backslash$AA$\backslash$ on dates 1862 and 2210 days after the supernova outburst. Even these pre-COSTAR observations avoid much of the contamination from the bright stars nearby and provide a very useful set of line strengths and shapes for analysis. The spectrum is formed in an unusual physical setting: cold gas which is excited and ionized by energetic electrons from the radioactive debris of the supernova explosion. The spectra of SN 1987A at this phase are surprisingly similar to those of the nova shells of CP Puppis and T Pyxidis decades after outburst. SN 1987A and the novae are characterized by emission from material with electron temperatures of only a few hundred degrees Kelvin, and show narrow Balmer continuum emission and strong emission lines from O\$\^{}+\$. The Balmer continuum shape requires the electron temperature in the supernova ejecta to be as low as 500 K on day 1862 and 400 K on day 2210 after outburst. The $\backslash$OIIUV$\backslash$ doublet is surprisingly strong and is plausibly powered by collisional ionization of neutral oxygen to excited states of O\$\^{}+\$. The line intensity ratio of the $\backslash$OID$\backslash$ doublet obtained from Gaussian fits of the line profiles is 1.8\$\backslash pm0.2\$, contrary to the optically thin limit of 3. This ratio is \{$\backslash$it not\} due to an optical depth effect, but rather is an artifact of assuming a Gaussian profile to fit the $\backslash$OID$\backslash$ doublet profile. Specifying the line ratio \$R\backslash, = \backslash, F([\{\backslash rm OI\}]6300)/F([\{\backslash rm OI\}]6364)\$ = 3 is consistent with the data and allows a calculation of the decomposed line profile. All the observed strong lines are found to be blueshifted by a similar amount},
archivePrefix = {arXiv},
arxivId = {astro-ph/9602157},
author = {Wang, Lifan and Wheeler, J. Craig and Kirshner, Robert P. and Challis, Peter M. and Filippenko, Alexei V. and Fransson, Claes and Panagia, Nino and Phillips, Mark M. and Suntzeff, Nicholas},
doi = {10.1086/177570},
eprint = {9602157},
file = {:Users/antoniabevan/Documents/Mendeley Desktop/Wang et al/\apj/Wang et al. - 1996 - Hubble Space Telescope Spectroscopic Observations of the Ejecta of SN 1987A at 2000 Days.pdf:pdf},
issn = {0004-637X},
journal = {\apj},
pages = {998--1010},
primaryClass = {astro-ph},
title = {{Hubble Space Telescope Spectroscopic Observations of the Ejecta of SN 1987A at 2000 Days}},
url = {http://arxiv.org/abs/astro-ph/9602157},
volume = {466},
year = {1996}
}
@article{Danziger2005,
author = {Danziger, I.J.},
file = {:Users/antoniabevan/Documents/Mendeley Desktop/Danziger/\memsai/Danziger - 2005 - Line profiles as diagnostics in supernova envelopes.pdf:pdf},
journal = {\memsai},
pages = {82},
title = {{Line profiles as diagnostics in supernova envelopes}},
volume = {7},
year = {2005}
}
@article{Zubko1996,
abstract = {The first results on the optical constants of three different amorphous carbon samples, possible analogues of interstellar and circumstellar dust grains, are presented. They have been deduced from recent laboratory data, making use of the Kramers-Kronig approach, It is shown that the physically correct simulation of clustering by the traditional `CDE' model is not applicable here. A modified CDE (MCDE) model is proposed and used in the present calculations. The MCDE model allows one to take into account the effect of percolation of the analysed amorphous carbon clusters through the parameter g(o), interpreted as a percolation strength, The intervals of the probable MCDE models (g(o)) have been chosen on the basis of the relevant estimates of Stognienko, Henning \& Ossenkopf for the cluster-cluster aggregation (CCA) and particle-cluster aggregation (PCA) models. The `optimal' percolation strengths and optical constants have been derived by using a least-squares procedure.},
author = {Zubko, V G and Mennella, V and Colangeli, L and Bussoletti, E},
file = {:Users/antoniabevan/Documents/Mendeley Desktop/Zubko et al/Mon. Not. R. Astron. Soc. Lett/Zubko et al. - 1996 - Optical constants of cosmic carbon analogue grains -- 1. Simulation of clustering by a modified continuous distrib.pdf:pdf},
issn = {0035-8711},
journal = {\mnras},
keywords = {carbon,circumstellar matter,dust,extinction,stars},
number = {4},
pages = {L1321--1329},
title = {{Optical constants of cosmic carbon analogue grains -- 1. Simulation of clustering by a modified continuous distribution of ellipsoids}},
volume = {282},
year = {1996}
}
@article{Groeningsson2006,
abstract = {High resolution spectra with UVES/VLT of SN 1987A from December 2000 until November 2005 show a number of high ionization lines from gas with velocities of roughly 350 km/s, emerging from the shocked gas formed by the ejecta-ring collision. These include coronal lines from [Fe X], [Fe XI] and [Fe XIV] which have increased by a factor of about 20 during the observed period. The evolution of the lines is similar to that of the soft X-rays, indicating that they arise in the same component. The line ratios are consistent with those expected from radiative shocks with velocity 310-390 km/s, corresponding to a shock temperature of (1.6-2.5) x 10\^{}6 K. A fraction of the coronal emission may, however, originate in higher velocity adiabatic shocks.},
archivePrefix = {arXiv},
arxivId = {astro-ph/0603815},
author = {Gr\"{o}ningsson, Per and Fransson, Claes and Lundqvist, Peter and Nymark, Tanja and Lundqvist, Natalia and Chevalier, Roger and Leibundgut, Bruno and Spyromilio, Jason},
doi = {10.1051/0004-6361:20065325},
eprint = {0603815},
file = {:Users/antoniabevan/Documents/Mendeley Desktop/Gr\"{o}ningsson et al/\aap/Gr\"{o}ningsson et al. - 2006 - Coronal emission from the shocked circumstellar ring of SN 1987A.pdf:pdf},
issn = {0004-6361},
journal = {\aap},
keywords = {circumstellar matter,individual,shock waves,sn 1987a,supernovae},
number = {5325},
pages = {11},
primaryClass = {astro-ph},
title = {{Coronal emission from the shocked circumstellar ring of SN 1987A}},
url = {http://arxiv.org/abs/astro-ph/0603815},
volume = {5325},
year = {2006}
}
@article{Mathis1977,
author = {Mathis, J. S. and Rumpl, W. and Nordsieck, K. H.},
doi = {10.1086/155591},
file = {:Users/antoniabevan/Documents/Mendeley Desktop/Mathis, Rumpl, Nordsieck/\apj/Mathis, Rumpl, Nordsieck - 1977 - The size distribution of interstellar grains.pdf:pdf},
issn = {0004-637X},
journal = {\apj},
month = oct,
pages = {425},
title = {{The size distribution of interstellar grains}},
url = {http://adsabs.harvard.edu/doi/10.1086/155591},
volume = {217},
year = {1977}
}
@article{Maeda2003,
abstract = {The light curves of 'hypernovae', i.e. very energetic supernovae with \$E\_\{51\} \backslash equiv E/10\^{}\{51\}\$ergs \$\backslash gsim 5-10\$ are characterized at epochs of a few months by a phase of linear decline. Classical, one-dimensional explosion models fail to simultaneously reproduce the light curve near peak and at the linear decline phase. The evolution of these light curves may however be explained by a simple model consisting of two concentric components. The outer component is responsible for the early part of the light curve and for the broad absorption features observed in the early spectra of hypernovae, similar to the one-dimensional models. In addition, a very dense inner component is added, which reproduces the linear decline phase in the observed magnitude-versus-time relation for SNe 1998bw, 1997ef, and 2002ap. This simple approach does contain one of the main features of jet-driven, asymmetric explosion models, namely the presence of a dense core. Although the total masses and energies derived with the two-component model are similar to those obtained in previous studies which also adopted spherical symmetry, this study suggests that the ejecta are aspherical, and thus the real energies and masses may deviate from those derived assuming spherical symmetry. The supernovae which were modeled are divided into two groups, according to the prominence of the inner component: the inner component of SN 1997ef is denser and more \$\^{}\{56\}\$Ni-rich, relative to the outer component, than the corresponding inner components of SNe 1998bw and 2002ap. These latter objects have a similar inner-to-outer component ratio, although they have very different global values of mass and energy.},
archivePrefix = {arXiv},
arxivId = {astro-ph/0305182},
author = {Maeda, K. and Mazzali, P. a. and Deng, J. and Nomoto, K. and Yoshii, Y. and Tomita, H. and Kobayashi, Y.},
doi = {10.1086/376591},
eprint = {0305182},
file = {:Users/antoniabevan/Documents/Mendeley Desktop/Maeda et al/\apj/Maeda et al. - 2003 - A Two-Component Model for the Light Curves of Hypernovae.pdf:pdf},
issn = {0004-637X},
journal = {\apj},
pages = {931--940},
primaryClass = {astro-ph},
title = {{A Two-Component Model for the Light Curves of Hypernovae}},
url = {http://arxiv.org/abs/astro-ph/0305182},
volume = {593},
year = {2003}
}
@article{Smith2012,
abstract = {Type IIn SNe show spectral evidence for strong interaction between their blast wave and dense circumstellar material (CSM) around the progenitor star. SN2010jl was the brightest core-collapse SN in 2010, and it was a Type IIn explosion with strong CSM interaction. Andrews et al. recently reported evidence for an IR excess in SN2010jl, indicating either new dust formation or the heating of CSM dust in an IR echo. Here we report multi-epoch spectra of SN2010jl that reveal the tell-tale signature of new dust formation: emission-line profiles becoming systematically more blueshifted as the red side of the line is blocked by increasing extinction. The effect is seen clearly in the intermediate-width (400--4000 km/s) component of H\$\backslash alpha\$ beginning roughly 30d after explosion. Moreover, we present near-IR spectra demonstrating that the asymmetry in the hydrogen-line profiles is wavelength dependent, appearing more pronounced at shorter wavelengths. This evidence suggests that new dust grains had formed quickly in the post-shock shell of SN 2010jl arising from CSM interaction. Since the observed dust temperature has been attributed to an IR echo and not to new dust, either (1) IR excess emission at \$\backslash lambda < 5 \backslash mu\$m is not a particularly sensitive tracer of new dust formation in SNe, or (2) some assumptions about expected dust temperatures might require further study. Lastly, we discuss one possible mechanism other than dust that might lead to increasingly blueshifted line profiles in SNeIIn, although the wavelength dependence of the asymmetry argues against this hypothesis in the case of SN2010jl.},
archivePrefix = {arXiv},
arxivId = {1108.2869},
author = {Smith, Nathan and Silverman, Jeffrey M. and Filippenko, Alexei V. and Cooper, Michael C. and Matheson, Thomas and Bian, Fuyan and Weiner, Benjamin J. and Comerford, Julia M.},
doi = {10.1088/0004-6256/143/1/17},
eprint = {1108.2869},
file = {:Users/antoniabevan/Documents/Mendeley Desktop/Smith et al/\aj/Smith et al. - 2012 - Systematic Blueshift of Line Profiles in the Type IIn Supernova 2010jl Evidence for Post-Shock Dust Formation.pdf:pdf},
issn = {0004-6256},
journal = {\aj},
keywords = {circumstellar matter,dust,evolution,extinction,mass-loss,outflows,stars,winds},
pages = {6},
title = {{Systematic Blueshift of Line Profiles in the Type IIn Supernova 2010jl: Evidence for Post-Shock Dust Formation?}},
url = {http://arxiv.org/abs/1108.2869},
volume = {17},
year = {2012}
}
@article{Ercolano2007,
abstract = {We present a study of the effects of clumping on the emergent spectral energy distribution (SED) from dusty supernova (SN) shells illuminated by a diffuse radiation source distributed throughout the medium. (...) The fully 3D radiation transport problem is solved using a Monte Carlo code, MOCASSIN, and we present a set of models aimed at investigating the sensitivity of the SEDs to various clumping parameters. We find that, contrary to the predictions of analytical prescriptions, the combination of an optical and IR observational data set is sufficient to constrain dust masses even in the case where optically thick clumps are present. Using both smoothly varying and clumped grain density distributions, we obtain new estimates for the mass of dust condensed by the Type II SN 1987A by fitting the optical and infrared spectrophotometric data of Wooden et al. (1993) at two epochs (day 615 and day 775). (...) From our numerical models we derive dust masses for SN 1987A that are comparable to previous analytic clumped graphite grain mass estimates, and at least two orders of magnitude below the 0.1-0.3 Msol that have been predicted to condense as dust grains in primordial core collapse supernova ejecta. This low condensation efficiency for SN 1987A is in contrast to the case of SN 2003gd, for which a dust condensation efficiency as large as 0.12 has recently been estimated. (Abridged)},
archivePrefix = {arXiv},
arxivId = {astro-ph/0611719},
author = {Ercolano, B. and Barlow, M. J. and Sugerman, B. E K},
doi = {10.1111/j.1365-2966.2006.11336.x},
eprint = {0611719},
file = {:Users/antoniabevan/Documents/Mendeley Desktop/Ercolano, Barlow, Sugerman/\mnras/Ercolano, Barlow, Sugerman - 2007 - Dust yields in clumpy supernova shells SN 1987A revisited.pdf:pdf},
issn = {00358711},
journal = {\mnras},
keywords = {Supernovae: individual: SN 1987A},
number = {3},
pages = {753--763},
primaryClass = {astro-ph},
title = {{Dust yields in clumpy supernova shells: SN 1987A revisited}},
volume = {375},
year = {2007}
}
@article{Fabbri2011,
author = {Fabbri, J. and Otsuka, M. and Barlow, M. J. and Gallagher, Joseph S. and Wesson, R. and Sugerman, B. E. K. and Clayton, Geoffrey C. and Meixner, M. and Andrews, J. E. and Welch, D. L. and Ercolano, B.},
doi = {10.1111/j.1365-2966.2011.19577.x},
file = {:Users/antoniabevan/Documents/Mendeley Desktop/Fabbri et al/\mnras/Fabbri et al. - 2011 - The effects of dust on the optical and infrared evolution of SN 2004et.pdf:pdf},
issn = {00358711},
journal = {\mnras},
keywords = {1 i n t,1989,2001,2003,and bianchi,circumstellar matter,ferrara,hasegawa,individual,nomoto,nozawa et al,ro d u c,schneider,sn 2004et,supernovae,t i o n,theoretical studies by kozasa,todini},
month = dec,
number = {2},
pages = {1285--1307},
title = {{The effects of dust on the optical and infrared evolution of SN 2004et}},
url = {http://doi.wiley.com/10.1111/j.1365-2966.2011.19577.x},
volume = {418},
year = {2011}
}
@misc{Phillips1990,
author = {Phillips, M. M. and Hamuy, M. and Heathcote, S. R. and Suntzeff, N. B. and Kirhakos, Sofia},
booktitle = {\aj},
doi = {10.1086/115402},
file = {:Users/antoniabevan/Documents/Mendeley Desktop/Phillips et al/\aj/Phillips et al. - 1990 - An optical spectrophotometric atlas of supernova 1987A in the LMC. II - CCD observations from day 198 to 805.pdf:pdf},
issn = {00046256},
pages = {1133},
title = {{An optical spectrophotometric atlas of supernova 1987A in the LMC. II - CCD observations from day 198 to 805}},
volume = {99},
year = {1990}
}
@techreport{Hanner1988,
author = {Hanner, M.},
title = {{Grain Optical Properties}},
year = {1988}
}
@article{Chugai1997,
author = {Chugai, Nikolai N. and Chevalier, Roger a. and Kirshner, Robert P. and Challis, Peter M.},
doi = {10.1086/304253},
file = {:Users/antoniabevan/Documents/Mendeley Desktop/Chugai et al/\apj/Chugai et al. - 1997 - Spectrum of SN 1987A at an Age of 8 Years Radioactive Luminescence of Cool Gas.pdf:pdf},
issn = {0004-637X},
journal = {\apj},
number = {2},
pages = {925--940},
title = {{Spectrum of SN 1987A at an Age of 8 Years: Radioactive Luminescence of Cool Gas}},
volume = {483},
year = {1997}
}
@article{Jerkstrand2012,
abstract = {SN 2004et is one of the nearest and best-observed Type IIP supernovae, with a progenitor detection as well as good photometric and spectroscopic observational coverage well into the nebular phase. Based on nucleosynthesis from stellar evolution/explosion models we apply spectral modeling to analyze its 140-700 day evolution from ultraviolet to mid-infrared. We find a MZAMS = 15 M⊙ progenitor star (with an oxygen mass of 0.8 M⊙) to satisfactorily reproduce [O i] $\lambda$$\lambda$6300, 6364 and other emission lines of carbon, sodium, magnesium, and silicon, while 12 M⊙ and 19 M⊙ models under- and overproduce most of these lines, respectively. This result is in fair agreement with the mass derived from the progenitor detection, but in disagreement with hydrodynamical modeling of the early-time light curve. From modeling of the mid-infrared iron-group emission lines, we determine the density of the "Ni-bubble" to $\rho$(t) ≃ 7 × 10-14 × (t/100 d)-3 g cm-3, corresponding to a filling factor of f = 0.15 in the metal core region (V = 1800 km s-1). We also confirm that silicate dust, CO, and SiO emission are all present in the spectra. Appendices are available in electronic form at http://www.aanda.org},
archivePrefix = {arXiv},
arxivId = {arXiv:1208.2183v1},
author = {Jerkstrand, a. and Fransson, C. and Maguire, K. and Smartt, S. and Ergon, M. and Spyromilio, J.},
doi = {10.1051/0004-6361/201219528},
eprint = {arXiv:1208.2183v1},
file = {:Users/antoniabevan/Documents/Mendeley Desktop/Jerkstrand et al/Astronomy \& Astrophysics/Jerkstrand et al. - 2012 - The progenitor mass of the Type IIP supernova SN 2004et from late-time spectral modeling.pdf:pdf},
issn = {0004-6361},
journal = {Astronomy \& Astrophysics},
keywords = {formation,general,identification,individual,line,radiative transfer,sn 2004et,supernovae},
pages = {A28},
title = {{The progenitor mass of the Type IIP supernova SN 2004et from late-time spectral modeling}},
volume = {546},
year = {2012}
}
@article{Tziamtzis2010,
abstract = {We used low resolution spectroscopy from VLT/FORS1, and high resolution spectra from VLT/UVES, to estimate the physical conditions in the ORs, using nebular analysis for emission lines such as [O II], [O III], [N II], and [S II]. We also measured the velocity at two positions of the ORs to test a geometrical model for the rings. Additionally, we used data from the HST science archives to check the evolution of the ORs of SN 1987A for a period that covers almost 11 years. We measured the flux in four different regions, two for each outer ring. We chose regions away from the two bright foreground stars, and as far as possible from the ER, and we created light curves for the emission lines of [O III], H\{$\backslash$alpha\}, and [N II]. The profiles of the lightcurves display a declining behaviour, which is consistent with the initial supernova flash powering of the ORs. The electron density of the emitting gas in the ORs, as estimated through nebular analysis from forbidden [O II] and [S II] lines, is <= 3*10\^{}3 cm\^{}-3, has not changed over the last \~{} 15 years, and also the [N II] temperature remains fairly constant at \~{} 1.2*10\^{}4 K. We find no obvious difference in density and temperature for the two ORs. The highest density, as measured from the decay of H\{$\backslash$alpha\}, could, however, be \~{} 5*10\^{}3 cm\^{}-3, and since the decay is somewhat faster for the southern outer ring than for the northern, the highest density in the ORs may be found in the southern outer ring. For an assumed distance of 50 kpc to the supernova, the distance between the supernova and the closest parts of the ORs could be as short as \~{} 1.7*10\^{}18 cm. Interaction between the supernova ejecta and the outer rings could therefore start in less than \~{} 20 years. We do not expect the ORs to show the same optical display as the equatorial ring when this happens.},
archivePrefix = {arXiv},
arxivId = {1008.3387},
author = {Tziamtzis, a. and Lundqvist, P. and Groningsson, P. and Nasoudi-Shoar, S.},
eprint = {1008.3387},
file = {:Users/antoniabevan/Documents/Mendeley Desktop/Tziamtzis et al/\aap/Tziamtzis et al. - 2010 - The outer rings of SN1987A.pdf:pdf},
journal = {\aap},
keywords = {individual objects,ism,line,profiles,sn 1987a,supernova remnants},
pages = {15},
title = {{The outer rings of SN1987A}},
url = {http://arxiv.org/abs/1008.3387},
volume = {35},
year = {2010}
}
@article{Matsuura2011,
abstract = {We report far-infrared and submillimeter observations of supernova 1987A, the star whose explosion was observed on 23 February 1987 in the Large Magellanic Cloud, a galaxy located 160,000 light years away. The observations reveal the presence of a population of cold dust grains radiating with a temperature of about 17 to 23 kelvin at a rate of about 220 times the luminosity of the Sun. The intensity and spectral energy distribution of the emission suggest a dust mass of about 0.4 to 0.7 times the mass of the Sun. The radiation must originate from the supernova ejecta and requires the efficient precipitation of all refractory material into dust. Our observations imply that supernovae can produce the large dust masses detected in young galaxies at very high redshifts.},
author = {Matsuura, M and Dwek, E and Meixner, M and Otsuka, M and Babler, B and Barlow, M J and Roman-Duval, J and Engelbracht, C and Sandstrom, K and Laki\'{c}evi\'{c}, M and van Loon, J Th and Sonneborn, G and Clayton, G C and Long, K S and Lundqvist, P and Nozawa, T and Gordon, K D and Hony, S and Panuzzo, P and Okumura, K and Misselt, K a and Montiel, E and Sauvage, M},
doi = {10.1126/science.1205983},
file = {:Users/antoniabevan/Documents/Mendeley Desktop/Matsuura et al/\sci/Matsuura et al. - 2011 - Herschel detects a massive dust reservoir in supernova 1987A.pdf:pdf},
issn = {1095-9203},
journal = {\sci},
month = sep,
number = {6047},
pages = {1258--61},
pmid = {21737700},
title = {{Herschel detects a massive dust reservoir in supernova 1987A.}},
url = {http://www.ncbi.nlm.nih.gov/pubmed/21737700},
volume = {333},
year = {2011}
}
@article{Gall2014,
abstract = {The origin of dust in galaxies is still a mystery. The majority of the refractory elements are produced in supernova explosions, but it is unclear how and where dust grains condense and grow, and how they avoid destruction in the harsh environments of star-forming galaxies. The recent detection of 0.1 to 0.5 solar masses of dust in nearby supernova remnants suggests in situ dust formation, while other observations reveal very little dust in supernovae in the first few years after explosion. Observations of the spectral evolution of the bright SN 2010jl have been interpreted as pre-existing dust, dust formation or no dust at all. Here we report the rapid (40 to 240 days) formation of dust in its dense circumstellar medium. The wavelength-dependent extinction of this dust reveals the presence of very large (exceeding one micrometre) grains, which resist destruction. At later times (500 to 900 days), the near-infrared thermal emission shows an accelerated growth in dust mass, marking the transition of the dust source from the circumstellar medium to the ejecta. This provides the link between the early and late dust mass evolution in supernovae with dense circumstellar media.},
author = {Gall, Christa and Hjorth, Jens and Watson, Darach and Dwek, Eli and Maund, Justyn R and Fox, Ori and Leloudas, Giorgos and Malesani, Daniele and Day-Jones, Avril C},
doi = {10.1038/nature13558},
file = {:Users/antoniabevan/Documents/Mendeley Desktop/Gall et al/\nat/Gall et al. - 2014 - Rapid formation of large dust grains in the luminous supernova 2010jl.pdf:pdf},
issn = {1476-4687},
journal = {\nat},
month = jul,
number = {7509},
pages = {326--9},
pmid = {25030169},
title = {{Rapid formation of large dust grains in the luminous supernova 2010jl.}},
url = {http://www.ncbi.nlm.nih.gov/pubmed/25030169},
volume = {511},
year = {2014}
}
@article{Owen2015,
author = {Owen, P. J. and Barlow, M. J.},
doi = {10.1088/0004-637X/801/2/141},
file = {:Users/antoniabevan/Documents/Mendeley Desktop/Owen, Barlow/\apj/Owen, Barlow - 2015 - the Dust and Gas Content of the Crab Nebula.pdf:pdf},
issn = {1538-4357},
journal = {\apj},
keywords = {ISM: individual objects (Crab Nebula),ISM: supernova remnants,circumstellar matter,crab nebula,individual objects,ism,supernova remnants},
number = {2},
pages = {141},
publisher = {IOP Publishing},
title = {{the Dust and Gas Content of the Crab Nebula}},
url = {http://stacks.iop.org/0004-637X/801/i=2/a=141?key=crossref.aebef8971f1fd2eeb2c2c6a485cadb9e},
volume = {801},
year = {2015}
}
@article{Groeningsson2007,
abstract = {We discuss high resolution VLT/UVES observations (FWHM \~{} 6 km/s) from October 2002 (day \~{}5700 past explosion) of the shock interaction of SN 1987A and its circumstellar ring. A nebular analysis of the narrow lines from the unshocked gas indicates gas densities of (1.5-5.0)E3 cm-3 and temperatures of 6.5E3-2.4E4 K. This is consistent with the thermal widths of the lines. From the shocked component we observe a large range of ionization stages from neutral lines to [Fe XIV]. From a nebular analysis we find that the density in the low ionization region is 4E6-1E7 cm-3. There is a clear difference in the high velocity extension of the low ionization lines and that of lines from [Fe X-XIV], with the latter extending up to \~{} -390 km/s in the blue wing for [Fe XIV], while the low ionization lines extend to typically \~{} -260 km/s. For H-alpha a faint extension up to \~{} -450 km/s can be seen probably arising from a small fraction of shocked high density clumps. We discuss these observations in the context of radiative shock models, which are qualitatively consistent with the observations. A fraction of the high ionization lines may originate in gas which has yet not had time to cool down, explaining the difference in width between the low and high ionization lines. The maximum shock velocities seen in the optical lines are \~{} 510 km/s. We expect the maximum width of especially the low ionization lines to increase with time.},
archivePrefix = {arXiv},
arxivId = {astro-ph/0703788},
author = {Gr\"{o}ningsson, Per and Fransson, Claes and Lundqvist, Peter and Lundqvist, Natalia and Leibundgut, Bruno and Spyromilio, Jason and Chevalier, Roger a. and Gilmozzi, Roberto and Kjaer, Karina and Mattila, Seppo and Sollerman, Jesper},
doi = {10.1051/0004-6361:200810551},
eprint = {0703788},
file = {:Users/antoniabevan/Documents/Mendeley Desktop/Gr\"{o}ningsson et al/\aap/Gr\"{o}ningsson et al. - 2007 - High resolution spectroscopy of the line emission from the inner circumstellar ring of SN 1987A and its hot.pdf:pdf},
issn = {0004-6361},
journal = {\aap},
keywords = {circumstellar matter,individual,shock waves,sn 1987a,supernovae},
pages = {19},
primaryClass = {astro-ph},
title = {{High resolution spectroscopy of the line emission from the inner circumstellar ring of SN 1987A and its hot spots}},
url = {http://arxiv.org/abs/astro-ph/0703788},
volume = {491},
year = {2007}
}
@article{Fransson2013,
abstract = {We present observations with VLT and HST of the broad emission lines from the inner ejecta and reverse shock of SN 1987A from 1999 until 2012 (days 4381 - 9100 after explosion). We detect broad lines from H-alpha, H-beta, Mg I], Na I, [O I], [Ca II] and a feature at 9220 A. We identify the latter line with Mg II 9218, 9244,most likely pumped by Ly-alpha fluorescence. H-alpha, and H-beta both have a centrally peaked component, extending to 4500 km/s and a very broad component extending to 11,000 km/s, while the other lines have only the central component. The low velocity component comes from unshocked ejecta, heated mainly by X-rays from the circumstellar ring collision, whereas the broad component comes from faster ejecta passing through the reverse shock. The reverse shock flux in H-alpha has increased by a factor of 4-6 from 2000 to 2007. After that there is a tendency of flattening of the light curve, similar to what may be seen in soft X-rays and in the optical lines from the shocked ring. The core component seen in H-alpha, [Ca II] and Mg II has experienced a similar increase, consistent with that found from HST photometry. The ring-like morphology of the ejecta is explained as a result of the X-ray illumination, depositing energy outside of the core of the ejecta. The energy deposition in the ejecta of the external X-rays illumination is calculated using explosion models for SN 1987A and we predict that the outer parts of the unshocked ejecta will continue to brighten because of this. We finally discuss evidence for dust in the ejecta from line asymmetries.},
archivePrefix = {arXiv},
arxivId = {arXiv:1212.5052v1},
author = {Fransson, Claes and Ergon, Mattias and Challis, Peter J. and Chevalier, Roger A. and France, Kevin and Kirshner, Robert P and Marion, G. H. and Milisavljevic, Dan and Smith, Nathan and Bufano, Filomena and Friedman, Andrew S. and Kangas, Tuomas and Larsson, Josefin and Mattila, Seppo and Benetti, Stefano and Chornock, Ryan and Czekala, Ian and Soderberg, Alicia and Sollerman, Jesper},
doi = {10.1088/0004-637X/797/2/118},
eprint = {arXiv:1212.5052v1},
file = {:Users/antoniabevan/Documents/Mendeley Desktop/Fransson et al/\apj/Fransson et al. - 2014 - HIGH-DENSITY CIRCUMSTELLAR INTERACTION IN THE LUMINOUS TYPE IIn SN 2010jl THE FIRST 1100 DAYS.pdf:pdf},
issn = {1538-4357},
journal = {\apj},
month = dec,
number = {2},
pages = {118},
title = {{HIGH-DENSITY CIRCUMSTELLAR INTERACTION IN THE LUMINOUS TYPE IIn SN 2010jl: THE FIRST 1100 DAYS}},
url = {http://arxiv.org/abs/1212.5052v1$\backslash$npapers://d3978ab5-8702-4bf9-be6f-7c3f6904d049/Paper/p11223$\backslash$nhttp://adsabs.harvard.edu/abs/2013ApJ...768...88F$\backslash$nhttp://stacks.iop.org/0004-637X/768/i=1/a=88?key=crossref.b349269625e16caea450d028f78a7c4e http://stacks.iop.o},
volume = {797},
year = {2014}
}
@inproceedings{Lucy1989,
author = {Lucy, L and Danziger, I and Gouiffes, C and Bouchet, P},
booktitle = {IAU Colloq. 120: Structure and Dynamics of the Interstellar Medium, Lecture Notes in Physics},
doi = {10.1007/BFb0114861},
editor = {Tenorio-Tagle, G. and Moles, M. and Melnick, J.},
file = {:Users/antoniabevan/Documents/Mendeley Desktop/Lucy et al/IAU Colloq. 120 Structure and Dynamics of the Interstellar Medium, Lecture Notes in Physics/Lucy et al. - 1989 - Dust condensation in the ejecta of SN 1987A.pdf:pdf},
number = {350},
pages = {164},
publisher = {Berlin Springer Verlag},
title = {{Dust condensation in the ejecta of SN 1987A}},
year = {1989}
}
@article{Gerasimovic1933,
author = {Gerasimovic, B.P.},
file = {:Users/antoniabevan/Documents/Mendeley Desktop/Gerasimovic/Zeitschrift f\"{u}r Astrophysik/Gerasimovic - 1933 - The contours of emission lines in expanding nebular envelopes.pdf:pdf},
journal = {Zeitschrift f\"{u}r Astrophysik},
pages = {335},
title = {{The contours of emission lines in expanding nebular envelopes}},
volume = {7},
year = {1933}
}
@article{Spyromilio1991,
author = {Spyromilio, J and Stathakis, R.A. and Cannon, R.D. and Waterman, L. and Couch, W.J. and Dopita, M.A.},
file = {:Users/antoniabevan/Documents/Mendeley Desktop/Spyromilio et al/\mnras/Spyromilio et al. - 1991 - Optical spectroscopy of supernova 1987 A at the AAT. I - The FORS data.pdf:pdf},
journal = {\mnras},
pages = {465},
title = {{Optical spectroscopy of supernova 1987 A at the AAT. I - The FORS data}},
url = {http://adsabs.harvard.edu/full/1991MNRAS.248..465S},
volume = {248},
year = {1991}
}
@article{Li1992,
abstract = {Model calculations are presented to fit the evolution of the intensity and profile of the forbidden O I 6300, 6364 emission doublet observed in the spectrum of SN 1987A. The fact that the doublet ratio R = F(6300)/F(6364) is less than 3 is evidence that these lines are optically thick for t of approximately less than 1 yr after outburst. The oxygen is theorized to be distributed in a clumpy fashion throughout the supernova envelope. The data is fitted with a three-zone model having 1.3 solar mass of oxygen distributed within a freely expanding sphere with a volume fitting factor f of about 0.1. The temperature of the oxygen changed from about 4100 K at day 200 to about 2900 K at day 500.},
author = {Li, Hongwei and McCray, Richard},
doi = {10.1086/171082},
file = {:Users/antoniabevan/Documents/Mendeley Desktop/Li, McCray/\apj/Li, McCray - 1992 - The forbidden O I 6300, 6364 A doublet of SN 1987A.pdf:pdf},
issn = {0004-637X},
journal = {\apj},
pages = {309},
title = {{The forbidden O I 6300, 6364 A doublet of SN 1987A}},
volume = {387},
year = {1992}
}
